\section{Conclusi'on}

Organizar la informaci'on utilizando ABBs balanceados permite acceder a ella eficientemente. Este r'apido acceso se torna imprescindible cuando el vol'umen de informaci'on que debemos manejar es enorme. Sin embargo, lograr el balanceo no es una tarea sencilla, y para conseguirlo necesitamos imponer ciertas restricciones que sean sostenibles a lo largo de las modificaciones que se efect'uen sobre el 'arbol.

Hemos explorado distintos criterios para forzar el balanceo de un 'arbol, observando que algunos de ellos son demasiado r'igidos como para poder ser mantenidos. El balanceo en altura prob'o ser efectivo en ese sentido. Queda claro que podr'ia haber otras alternativas para llegar al preciado balanceo, y de hecho las hay, como por ejemplo aquellas utilizadas por los 'arboles red-black, 'arboles B o 'arboles 2-3-4. Si bien todas estas estructuras cumplen con el objetivo de almacenar en orden y en forma balanceada toda la informaci'on, no todos son igualmente complejos: la distinta rigidez entre los criterios de balanceo que utilizan hace que la dificultad de los algoritmos de cada estructura y, a nivel de eficiencia, la cantidad de rebalanceos que necesitan, var'ie entre ellas. El balanceo de 'arboles, por lo tanto, no es s'olo un concepto utilizado por los 'arboles AVL, sino que tiene la jerarqu'ia de un campo de estudio en computaci'on.