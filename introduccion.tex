%\section{Disclaimer}

%Este apunte se provee como una ayuda adicional; un repaso en castellano de la bibliografía sobre ABBs y AVLs y de las respectivas teóricas. Su lectura no es un reemplazo de ninguna de las anteriores. Si bien es una versi'on estable, est'a en proceso de depuración (agradeceremos reportar al autor cualquier error o sugerencia). Ante cualquier duda, rogamos remitirse a la bibliografía oficial de la materia.

\section{Introducci'on}

La estructura de 'arbol es una idea fundamental en la computaci'on. Su importancia radica en su capacidad para estructurar informaci'on de manera din'amica, es decir, posibilitando  la f'acil realizaci'on de cambios en esa informaci'on a lo largo del tiempo. El concepto de balanceo aparece como una soluci'on al crecimiento desmesurado de la altura de los mismos, lo cual imposibilita realizar operaciones que requieran atravesar el 'arbol desde la ra'iz hasta una hoja, en forma eficiente.

En este trabajo revisamos la noci'on de balanceo, y estudiamos diversas familias de 'arboles que son balanceados. Una de estas familias derivar'a en el concepto de 'arbol AVL. Estos 'arboles son balanceados y permiten ejecutar eficientemente las operaciones de diccionario, lo cual los hace una representaci'on perfecta para estos tipos abstractos de datos. Por esta raz'on, en la segunda parte del trabajo nos concentramos en la implementaci'on detallada de las operaciones de un AVL.